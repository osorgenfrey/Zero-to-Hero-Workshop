\section{Variables}
The purpose of these exercises is to provide students with a better understanding of what happens when you create and assign values to variables. The exercises cover creating variables, copying one variable to another, and swapping the values of two variables.

\subsection{Introduction}
What is computational thinking?\\
When programming it is important to understand how a computer works. This is called computational thinking. With our workshops we try to convey how to a computer thinks, and therefore make your life easier. 

What are Variables?\\
A variable is a way to save something in the computer, for it to be used again later. If we need the computer to remember a number, a word, or something more complex as a list, we save it in a variable. 

What are we going to do today?\\
We will go through how variables are created and the rules of what happens to them when you change them. 

\subsection*{How to initiate}
\begin{itemize}
    \item \textit{"I see you're having issues with your variables. Would you like an overview of how they work so that you can solve the problem?"}
    \item \textit{"Oh this is a variable problem. Do you want to know how they work?"}
    \item \textit{"OK put the laptop aside and let's try this instead"}
    \item (For the smart student) \textit{"You're very familiar with the building blocks of code, could we try this together, and review if it makes sense?"}
\end{itemize}


\subsection*{Materials}
\begin{itemize}
    \item[-] Boxes that can fold in on themselves and lay flat, with labels "A", "B", and "C".
    \item[-] "Items" in the form of printed out bananas and apples.
    \item[-] List of the rules of the variables.
\end{itemize}

\subsection*{The Main Rules}
You must explain the rules of our "variables" before beginning the exercises.
\begin{itemize}
    \item[-] When a box is opened (variable created), it cannot be empty
    \item[-] There can only be one item per box.
    \item[-] Names of variables on the left
    \item[-] Values of variables on the right
\end{itemize}



\subsection{Creating and copying}
You will give these instructions verbally and have the student perform them as you go. The purpose of this exercise is to help them understand how our "variables" work.

Explain that, first, we must create the variables.

\textbf{Creating Variable A}
\begin{itemize}
    \item[-] Pick up box A and unfold it
    \item[-] Put an item of your choosing in box A
\end{itemize}

\textbf{Creating Variable B}
\begin{itemize}
    \item[-] Pick up box B and unfold it
    \item[-] Put a different item in box B
\end{itemize}

\textbf{Copying B to A (A = B)}
\begin{itemize}
    \item[-] Put the same item that is in box B, in box A
\end{itemize}

Explain that, by copying one variable to the other, you "discard" the value that it was originally given. Show them the code snippet labeled "Exercise 1", and explain that this was what they just did.

\newpage
\subsection{Swapping two variables}
For the next task, you will guide the student through the code snippet labeled "Exercise 2". The code walks through how variables are swapped by creating a temporary extra variable. Show the code to the student and take it line by line, without telling the purpose of the task. If the student is confident enough to just follow the instructions without help, you should tell them to explain what they are doing and solidify their understanding of the values being discarded. When finished, follow through with an explanation of how they actually just swapped the values of the variables.

The worded instructions could look as follows:
\begin{itemize}
    \item[-] Create box \textbf{A} with a banana
    \item[-] Create box \textbf{B} with an apple
    \item[-] Create box \textbf{C} with a copy of \textbf{A} (\textbf{C} = \textbf{A})
    \item[-] Copy \textbf{B} to \textbf{A} (\textbf{A} = \textbf{B}) 
    \item[-] Copy \textbf{C} to \textbf{B} (\textbf{B} = \textbf{C})
\end{itemize}

% \subsection{\textbf{Hard} - Pseudocode}
% The purpose of this exercise is to give the student code snippets to perform and let them realize they are actually just repeating the former exercises.

% Upon finishing the tasks, you should help them make that conclusion, by asking along the lines of \textit{"Do you recognize what happened here?"}.


\subsection{Extra - What's in the box}
These exercises are for students, who are having a hard time understanding the outcome of manipulating the variables. For these, you should give them all instructions, ask them what they think the outcome will be, and then have them perform it hands on.

\textbf{1. }
\begin{itemize}
    \item[-] Put a banana in box \textbf{A}
	\item[-] Put in box \textbf{B} the same as in box \textbf{A}
    \item[-] Put an apple in box \textbf{A}
\end{itemize}

\textbf{2. }
\begin{itemize}
    \item[-] Put a banana in box \textbf{A}
    \item[-] Put an apple in box \textbf{B}
	\item[-] Put in box \textbf{A} the same as in box \textbf{B}
    \item[-] Put in box \textbf{B} the same as in box \textbf{A}
\end{itemize}

\newpage